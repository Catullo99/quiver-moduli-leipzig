\documentclass{beamer}
% Layout, characters, page margins, pictures handling.
\usepackage{fontenc}                                % Fonts
\usepackage{graphicx, float}                        % Images
\usepackage{geometry}    % Page layout, margins etc
% \usepackage{parskip}                                % To handle paragraph spacing, indentation etc. Not for beamers.
\usepackage{colonequals}

% Math tools
\usepackage{amsmath, amsthm, amsfonts, amssymb}     % A lot of symbols

% References and hyperlinks both in the document and to the web. Needs two compilations in a row.
\usepackage{hyperref}
\usepackage{bookmark}

% Commutative diagrams
\usepackage{tikz, tikz-cd}
\usetikzlibrary{calc} % This allows for slightly moving nodes in a tikzpicture, by using ($(m-1-1.east)+(offsetX,offsetY)$) for example.
\usetikzlibrary{arrows, matrix} % I guess


%%%%%%%%%%%%%%%%%%%%%%%%%%%%%%%%%%%%%%%%%%%%%%%%%%%%%%%%%%%%%%%%%%%%%%%%%%%%%%%%%%%%%%%%%%%%%%%%%%%
% Bibliography management. Compile with biber.
\usepackage[backend=biber, datamodel=mrnumber, sortcites]{biblatex}
\addbibresource{bibliography.bib}
\renewcommand*{\bibfont}{\tiny}


% \usepackage{gitinfo2}

% \newcommand\gitfootnote[1]{% yoinked from Pieter's repo
%   \begin{NoHyper}
%   \renewcommand\thefootnote{}\footnote{#1}%
%   \addtocounter{footnote}{-1}%
%   \end{NoHyper}
% }

%%%%%%%%%%%%%%%%%%%%%%%%%%%%%%%%%%%%%%%%%%%%%%%%%%%%%%%%%%%%%%%%%%%%%%%%%%%%%%%%%%%%%%%%%%%%%%%%%%%
% Theorems are numbered within sections using a common counter. Do away with ambiguous numbering schemes and assign A number to A thing.
% \theoremstyle{plain}
% \newtheorem{theorem}{Theorem}[section]
% \newtheorem{proposition}[theorem]{Proposition}
% \newtheorem{lemma}[theorem]{Lemma}
% \newtheorem{corollary}[theorem]{Corollary}
% \newtheorem{claim}[theorem]{Claim}

% \theoremstyle{remark}
% \newtheorem{remark}[theorem]{Remark}

% \theoremstyle{definition}
% \newtheorem{definition}[theorem]{Definition}
% \newtheorem{example}[theorem]{Example}



%%%%%%%%%%%%%%%%%%%%%%%%%%%%%%%%%%%%%%%%%%%%%%%%%%%%%%%%%%%%%%%%%%%%%%%%%%%%%%%%%%%%%%%%%%%%%%%%%%%%


\newcommand{\iif}{\ensuremath{\Leftrightarrow}}              % If and only if

\renewcommand*{\to}[1][]{\overset{#1}{\rightarrow}} % Arrow with optional label. Use as A \to[label] B

\newcommand{\dirlim}[1]{                        % Direct limit
    \varinjlim_{\substack{#1}}
    }

\newcommand{\invlim}[1]{                        % Inverse limit
    \varprojlim_{\substack{#1}}
    }

\DeclareMathOperator{\Hom}{Hom}
\DeclareMathOperator{\dom}{dom}
\DeclareMathOperator{\im}{Im}
\DeclareMathOperator{\GL}{GL}
\newcommand{\Mat}{\mathrm{Mat}} %matrix space


% Numbers
\newcommand{\NN}{\mathbb{N}}
\newcommand{\ZZ}{\mathbb{Z}}
\newcommand{\QQ}{\mathbb{Q}}
\newcommand{\RR}{\mathbb{R}}
\newcommand{\CC}{\mathbb{C}}

% Spaces
\renewcommand{\AA}{\mathbb{A}}
\newcommand{\PP}{\mathbb{P}}

% macros
\newcommand{\tuple}[1]{\mathbf{#1}}
\DeclareMathOperator{\parameterspace}{R}
\DeclareMathOperator{\modulispace}{M}
\DeclareMathOperator{\group}{\GL_{\tuple{d}}}

\newcommand{\stable}{-\mathrm{st}}
\newcommand{\semistable}{-\mathrm{sst}}

\DeclareMathOperator{\HH}{H}

\title{An introduction to quiver moduli}
\institute{Seminar on Nonlinear Algebra - MPI MIS Leipzig}
\author{Gianni Petrella - University of Luxembourg}
\date{June 20, 2024}

\begin{document}
\begin{frame}
    \titlepage
    \footnotetext{Slides at \href{github.com/Catullo99/quiver-moduli-leipzig}{github.com/Catullo99/quiver-moduli-leipzig}}
    % \gitfootnote{commit: \texttt{\gitAbbrevHash}\hfil date: \texttt{\gitAuthorIsoDate}\hfil \texttt{\gitReferences}}
\end{frame}
\begin{frame}
    \frametitle{An introduction to quiver moduli}
{\bf Plan}
    \begin{enumerate}
    \item Construction of moduli spaces of representations of quivers. \pause
    \item Descent of the universal family on quiver moduli. \pause
    \item The standard exact sequence and its consequences. \pause
    \item Rigidity for quiver moduli.
\end{enumerate} \pause
\vfill
{\bf Aknowledgements}

{\small This work is supported by the Luxembourg National Research Fund (AFR-17953441)}

{\small Joint work with P. Belmans, A. Brecan, H. Franzen, M. Reineke}
\end{frame}

\begin{frame}
    \frametitle{Quiver representations}
\begin{definition}
    We work over a field $k = \bar{k}$ of characteristic $0$. \pause

    A \emph{quiver}~$Q$ is a finite directed graph.

    A \emph{representation}~$V$ of~$Q$ over~$k$ is
    a choice of a vector space~$V_i$ per vertex~$i$
    and of a linear map~$V_{\alpha}$ per arrow~$\alpha$.
\end{definition} \pause

Two representations of the same \emph{dimension vector}~$V, W$
are isomorphic if there is a common base change~$M_i : V_i \to[\sim] W_i$
such that for all vertices~$i, j$ and all arrows~$\alpha : i \to j$,
\[M_j \circ V_{\alpha} = W_{\alpha}\circ M_i,~~~\text{that is,}~~~ \begin{tikzcd}[ampersand replacement=\&]
V_i \arrow[r, "V_{i \to j}"] \arrow[d, "M_i"] \& V_j \arrow[d, "M_j"] \\
W_i \arrow[r, "W_{i \to j}"] \& W_j.
\end{tikzcd} \]
\end{frame}

\begin{frame}
    \frametitle{Quiver moduli 1/2}
Once a dimension vector~$\tuple{d}$ is fixed,
a representation is determined by a point in the \emph{parameter space}
\[\parameterspace(Q, \tuple{d}) \colonequals \bigoplus_{i \to j \in Q_1} \Mat_{d_j \times d_i}(k).\]
\pause
Isomorphism classes are orbits of the action of
\[{\GL}_{\tuple{d}} \colonequals \bigoplus_{i \in Q_0} \GL_{d_i}(k),\]
which for $g = (g_i)_{i \in Q_0}$ and $M = (M_{a})_{a : i \to j \in Q_1}$ is defined by
\[g \cdot M \colonequals (g_j \cdot M_{a} \cdot g^{-1}_i).\] \pause
{\bf Problem: } The GIT quotient is just a bunch of points!
\end{frame}
\begin{frame}
    \frametitle{Quiver moduli 2/2}
A \emph{stability parameter} is~$\theta \in \ZZ^{Q_0}$
for which~$\theta \cdot \tuple{d} = 0$. \pause
\begin{definition}
The representation~$V$ is \emph{stable} (respectively \emph{semistable})
if all its proper subrepresentations~$W$ satisfy~$\theta \cdot W < 0$
(respectively~$\theta \cdot W \leq 0$).
\end{definition} \pause
\begin{theorem}
The stable locus~$\parameterspace^{\theta\stable}(Q, \tuple{d})$
(respectively the semistable locus~$\parameterspace^{\theta\semistable}(Q, \tuple{d})$)
is a~$\GL_{\tuple{d}}$-invariant Zariski open which admits a geometric quotient,
denoted by~$\modulispace^{\theta\stable}(Q, \tuple{d})$
(respectively by~$\modulispace^{\theta\semistable}(Q, \tuple{d})$).
\end{theorem} \pause

{\bf Facts}
\begin{enumerate}
    \item The GIT quotient of the semistable locus is projective. \pause
    \item The GIT quotient of the stable locus is smooth.
\end{enumerate}
\end{frame}
\begin{frame}
    \frametitle{``Holy diagram'' of quiver moduli}
    \[\begin{tikzcd}[row sep=small, column sep=small, ampersand replacement=\&]
        \parameterspace^{\theta\stable}(Q, \tuple{d}) \arrow[r, hook, "open"] \arrow[d, "\pi"]   \& \parameterspace^{\theta\semistable}(Q, \tuple{d}) \arrow[r, hook, "open"] \arrow[d, "\pi"]                                   \& \parameterspace(Q, \tuple{d}) \arrow[d, "\pi"] \\ 
        \modulispace^{\theta\stable}(Q, \tuple{d}) \arrow[r, hook, "open"] \arrow[d, equal]    \&  \modulispace^{\theta\semistable}(Q, \tuple{d}) \arrow[r] \arrow[d, equal]                                       \& \modulispace(Q, \tuple{d}) \arrow[d, equal] \\
        \parameterspace^{\theta\stable}(Q, \tuple{d}) /_\theta \GL_{\tuple{d}} \arrow[r, hook]           \&  Proj \left(\bigoplus_{n} \mathcal{O}_{\parameterspace(Q, \tuple{d})}^{n\theta}\right) \arrow[r]    \& Spec(\mathcal{O}_{\parameterspace(Q, \tuple{d})}^{\GL_{\tuple{d}}})
    \end{tikzcd} \] \pause
\begin{lemma}[Adriaenssens--Le Bruyn \cite{MR1972892}]
    The ring of invariants $\mathcal{O}^{\GL_{\tuple{d}}}_{\parameterspace(Q, \tuple{d})}$
    is generated by elements in bijection with oriented cycles in the quiver.
\end{lemma} \pause
\begin{corollary}
    If $Q$ is \emph{acyclic}, $\modulispace^{\theta\semistable}(Q, \tuple{d})$ is a projective variety.
\end{corollary} \pause
From now on, our quivers are acyclic.
\end{frame}

\begin{frame}
    \frametitle{Intermezzo: quiver moduli vs moduli of bundles on curves}
There are several analogies between moduli of quiver representations and of VBACs. \pause

\begin{itemize}
    \item Abelian categories of cohomological dimension 1;
    \item GIT construction of moduli space;
    \item GIT-free construction of moduli stack, descent of ample line bundle;
    \item Harder--Nararasimhan filtrations, slope stability;
    \item Semiorthogonal decompositions of derived categories...
\end{itemize} \pause

Along with birational classifications, Brauer groups...\pause

{\tiny ask me about these later if you are interested! :)}

\end{frame}
\begin{frame}
    \frametitle{Universal families 1/2}
{\small
\begin{definition}
    A \emph{universal family} for a moduli space $M$ is a $\mathcal{U} \to M$
    such that for any family $W \to B$ of the objects being parametrized over an arbitrary $B$,
    there exists a unique morphism $B \to M$ for which $W$ fits in a cartesian diagram
\[
    \begin{tikzcd}[ampersand replacement=\&]
        W \arrow[d] \arrow[r, dashed] \& \mathcal{U} \arrow[d] \\
        B \arrow[r, dashed, "\exists!"] \& M
    \end{tikzcd}
\]
\end{definition}} \pause
On $\parameterspace(Q, \tuple{d})$, let~$U_i \colonequals \parameterspace \times \mathbb{A}^{d_i}$
be trivial vector bundles. \pause

For every arrow $a : i \to j$, let $U_{a} : U_i \to U_j$ be a morphism of vector bundles that
sends $(x, p)$ to $(x, M_{x, a}(p))$. \pause

\begin{lemma}
For all $x \in \parameterspace(Q, \tuple{d})$,
the fiber of $U \colonequals \oplus_i U_i$ on $x$,
together with $\{U_a|_{x}~|~ a \in Q_1\}$,
is equal to the representation encoded by $x$.
\end{lemma}
\end{frame}

\begin{frame}
    \frametitle{Universal families 1.5/2}   
{\bf Question:} does this vector bundle descend to the quotient, i.e.,
is there a vector bundle on $\modulispace^{\theta\stable}(Q, \tuple{d})$
fitting in the diagram below? \pause
\[ \begin{tikzcd}[row sep=0.5cm, ampersand replacement=\&]
    \parameterspace^{\theta\stable}(Q, \tuple{d}) \times U_i \arrow[r]\arrow[d, dashed] \& \parameterspace^{\theta\stable}(Q, \tuple{d}) \arrow[d, "\pi"] \\ 
    {?} \arrow[r, dashed] \& \modulispace^{\theta\stable}(Q, \tuple{d})
\end{tikzcd} \] \pause

To \emph{descend}, $U$ must have the extra structure of
a~$\group$-action, and must satisfy the descent condition: \pause
\begin{lemma}[Kempf, Théorème 2.3 \cite{MR0999313}]
    Any equivariant vector bundle $E$ descends to $\modulispace^{\stable}$ if and only if
    for all $x \in R^{\theta\stable}$ the stabilizer of
    $x$ acts trivially on the fiber $E_x$.
\end{lemma}
\end{frame}

\begin{frame}
    \frametitle{Universal families 2/2}
\begin{definition}
The dimension vector $\tuple{d}$ is \emph{$\theta$-coprime} if
$\forall~ 0 \neq \tuple{e} \leq \tuple{d}$, $\theta \cdot \tuple{e} \neq 0$.
\end{definition} \pause
In particular, $\gcd(\tuple{d}) = 1$,
and a representation $V$ is $\theta$-semistable
if and only if it is $\theta$-stable, i.e., $\parameterspace^{\theta\semistable} = \parameterspace^{\theta\stable}$. \pause

\begin{lemma}
    The bundles $U_i$ descend if $\gcd(\tuple{d}) = 1$.
\end{lemma}\pause
\begin{proof}
Under the hypothesis, $\exists~\tuple{a} \in \ZZ^{Q_0}$
s.t. $\tuple{a} \cdot \tuple{d} = 1$. \pause

Then, there is an action of $\group$ on
$\parameterspace^{\theta\stable}(Q, \tuple{d}) \times U_i$ as
\[g \cdot (x, s) \colonequals (g \cdot x, \prod_{v \in Q_0} \det(g_v)^{-a_v} g_i \cdot s). \] \pause

The stabilizer of any stable representation $x$ is $\mathbb{G}_m$, which fixes $U_{i, x}$.
\end{proof} \pause

The resulting bundles on $\modulispace$ are denoted by $\mathcal{U}_i$.
% \begin{lemma}
%     This suffices to define an action of $G$ on $U_i$ for all $i$,
%     and the resulting equivariant structure descends. \pause
% \end{lemma}

\end{frame}
\begin{frame}
    \frametitle{The fundamental exact sequence}
The universal bundles $\mathcal{U}_i$ fit into a 4-terms exact sequence on $\modulispace^{\theta\stable}(Q, \tuple{d})$: \pause
\begin{equation}
\label{equation:4-term-exact-sequence}
0
\to\mathcal{O}_{\modulispace}
\to\bigoplus_{i \in Q_0} \mathcal{U}^\vee_i \otimes \mathcal{U}_i
\to\bigoplus_{a: i \to j \in Q_1} \mathcal{U}^\vee_{i} \otimes \mathcal{U}_{j}
\to\mathcal{T}_{\modulispace}
\to 0.
\end{equation} \pause

{\bf Applications include}
\begin{enumerate}
    \item Deformation theory \cite{belmans2023rigidity};
    \item Presentation of Chow rings \cite{Franzen2015} and intersection theory \cite{belmans2023chow};
    \item Understanding of some Brauer groups \cite{reineke2014brauer};
    \item Fano-ness of quiver moduli \cite{Franzen2020};
    \item ...
\end{enumerate}
\end{frame}
\begin{frame}
    \frametitle{Rigidity for quiver moduli 1/n}
\begin{theorem}[Belmans--Brecan--Franzen--P.--Reineke]
If $\tuple{d}$ is $\theta$-coprime\footnote{ plus some technical assumptions}, for any $i, j \in Q_0$ and for all $k \geq 1$ we have
\[\HH^{k}(\modulispace^{\theta\stable}(Q, \tuple{d}),~\mathcal{U}^{\vee}_{i} \otimes \mathcal{U}_j) = 0. \]
\end{theorem} \pause
\begin{corollary}
The variety $\modulispace^{\theta\stable}(Q, \tuple{d})$ is \emph{infinitesimally rigid}, i.e.,
\[ \HH^1(\modulispace^{\theta\stable}(Q, \tuple{d}), \mathcal{T}_{\modulispace}) = 0. \]
\end{corollary} \pause

{\bf Note:} the group $\HH^1(X, \mathcal{T}_{X})$ parametrizes infinitesimal
deformations of a projective variety $X$.
\end{frame}

\begin{frame}
    \frametitle{Proof background - GIT stratification}
We recall a stratification result of Geometric Invariant Theory. \pause
\begin{theorem}[Kempf--Ness, Hesselink]
Given $G$ a reductive algebraic group
acting on a (projective) variety $X$,
there is a stratification of $X$
into smooth, disjoint, locally closed subsets $S_{\ell}$,
ordered in a way such that $S_0 = X^{ss}$ and
\[\bar{S_{\ell}} \subset \cup_{m \geq \ell} S_m.\]
\end{theorem} \pause

Each stratum $S_{\ell}$ is $G$-invariant and is indexed by
a one-parameter subgroup of $G$,
that is, by an embedded copy of
$\lambda_{\ell} \colon \mathbb{G}_m \hookrightarrow G$. \pause

{\bf Notes} \pause
\begin{itemize}
    \item The fixed locus of $\lambda_{\ell}$ is denoted by $Z_{\ell} \subset S_{\ell}$. \pause
    %in fact there is a principal bundle structure
    %\[ S_i \times^{P(\lambda_i)} G \to Z_i.\] \pause
    \item The group $\lambda_{\ell}$ acts on $\det(\mathcal{N}_{S_{\ell}/X})|_{Z_{\ell}}$. \pause
    \item Both $\lambda_{\ell}$ and $S_{\ell}$ depend on the choice of a norm on $1-PS$s.
\end{itemize} \pause
How does this apply to quiver moduli?
\end{frame}

\begin{frame}
    \frametitle{Harder--Nararasimhan stratification}
    For quiver moduli, the choice of norm on 1-PSs is given by $\alpha \in \NN^{Q_0}_0$.
    Denote $\mu(x) \colonequals \frac{\theta \cdot x}{\alpha \cdot x}$ for any $x \in \ZZ^{Q_0}$. \pause

\begin{definition}
    Given $\theta$ and $\alpha$, every representation admits a unique \emph{Harder--Narasimhan filtration},
    i.e., $ 0 = V_0 \subsetneq V_1 \dots \subsetneq V_s = V$ such that every successive quotient
    $V_i/V_{i-1}$ is semistable.
\end{definition} \pause
The sequence of dimension vectors $\dim(V_1), \dim(V_2/V_1), \dots$ is
called the \emph{HN type} of $V$, and denoted by $\tuple{d}^* = (\tuple{d}^1,\dots,\tuple{d}^s)$. \pause

{\bf Fact: } for all $1 \leq t < m \leq s$, $\mu(\tuple{d}^{t}) > \mu(\tuple{d}^m)$.\pause
\begin{theorem}[Reineke \cite{MR1974891}]
Separating representations in $\parameterspace(Q, \tuple{d})$ by their HN type gives
a stratification into smooth, disjoint, locally closed subsets $S_{\tuple{d}^*}$,
ordered in a way such that $S_{(\tuple{d})} = X^{ss}$ and
\[\bar{S_{\tuple{d}^*}} \subset \cup_{\tuple{e}^* \geq \tuple{d}^*} S_{\tuple{e}^*}.\] \pause
This is called the Harder--Narasimhan stratification.
\end{theorem}
\end{frame}
\begin{frame}{Harder--Nararasimhan equals GIT}
What is the relation between these two stratifications? \pause
    \begin{theorem}[Hoskins \cite{MR3261979, MR3871820}]
        For the action of $\group$ on $\parameterspace(Q, \tuple{d})$,
        ``twisted by $\theta$'', given a norm on 1-PSs $\alpha$,
        the GIT stratification coincides with the Harder--Narasimhan
        stratification given by $\theta$ and $\alpha$.
    \end{theorem}\pause
    
    \begin{corollary}
        The 1-PS $\lambda$ corresponding to the HN type $\tuple{d}^*$ is given by
        \[ \lambda_i(z) = \operatorname{diag}\big( \underbrace{z^{k_1},\ldots,z^{k_1}}_{d_i^1 \text{ times}}; \underbrace{z^{k_2},\ldots,z^{k_2}}_{d_i^2 \text{ times}};\ldots; \underbrace{z^{k_s},\ldots,z^{k_s}}_{d_i^s \text{ times}} \big),\]
        where $k_t = \mu(\tuple{d}^t)$.
    \end{corollary}
    
\end{frame}


\begin{frame}
    \frametitle{Proof background - Teleman quantization}
Let $\eta_{\ell}$ be the weight of the action of $\lambda_{\ell}$ on
$\det(\mathcal{N}_{S_{\ell}/X})|_{Z_{\ell}}$. \pause

Assume that $X^{ss}/G$ is a geometric quotient. \pause

For a coherent $G$-sheaf $\mathcal{F}$ on $X$ 
such that $\mathcal{F}|_{X^{ss}}$ descends,
there is a standard morphism for all $k \geq 0$:
\[ \HH^k(X, \mathcal{F}) \supset \HH^{k}(X, \mathcal{F})^{G} \to \HH^{k}(X^{ss} // G, \mathcal{F}_{desc}).\] \pause

On each stratum $S_{\ell}$,
denote the set of $\lambda_{\ell}$-weights of $\mathcal{F}|_{Z_\ell}$
by $W(\lambda_{\ell}, \mathcal{F}|_{Z_\ell})$. \pause
\begin{theorem}[Teleman quantization \cite{MR3327537}]

If on each stratum $S_{\ell}$, the inequality $\max(W(\lambda_{\ell}, \mathcal{F})) < \eta_{\ell}$
holds, then there is an isomorphism
\[\HH^{k}(X, \mathcal{F})^{G} \to[\sim] \HH^{i}(X^{ss} // G, \mathcal{F}_{desc})\]
for all $k \geq 0$.
\end{theorem}

\end{frame}

\begin{frame}
    \frametitle{Proof of rigidity - setup}
We apply Teleman quantization to the quiver moduli setup. \pause

Remember: we let $\group$ act on $\parameterspace(Q, \tuple{d})$,
we fix a stability parameter $\theta$ and a norm on $1-PS$ $\alpha$. \pause
We assume that $\tuple{d}$ is $\theta$-coprime.

On each stratum $S_{\tuple{d}^*}$, the group $\lambda_{\tuple{d}^*}$
acts on $\det(\mathcal{N}_{S_{\tuple{d}^*}/\parameterspace})|_{Z_{\tuple{d}^*}}$, \pause
and on ${U}^{\vee}_i \otimes {U}_j$. \pause

If we can apply Teleman quantization, we win, because $\parameterspace(Q, \tuple{d})$ is affine, so
\[\HH^k(\parameterspace(Q, \tuple{d}), \mathcal{F}) = 0 ~\forall k \geq 1,\]
so we compute the weights we are interested in.
\end{frame}

\begin{frame}
    \frametitle{Proof of rigidity - weights}
\begin{lemma}[Corollary 3.18 \cite{belmans2023rigidity}]
The weight $\eta_{\tuple{d}^*}$ of $\det(\mathcal{N}_{S_{\tuple{d}^*}/\parameterspace})|_{Z_{\tuple{d}^*}}$ is
\[\eta_{\tuple{d}^*} = \sum_{1 \leq m < n \leq s} (k_n - k_m)\langle \tuple{d}^m, \tuple{d}^n \rangle.\]
\end{lemma} \pause

\begin{lemma}[Lemma 3.20 \cite{belmans2023rigidity}]
    The $\lambda_{\tuple{d}^*}$-weights of $(U^{\vee}_i \otimes {U}_j)|_{Z_{\tuple{d}^*}}$ are
    \[W(\lambda_{\tuple{d}^*}, U_i^{\vee} \otimes U_j) = \{k_m - k_n ~|~ 1 \leq m, n \leq s\}.\]
\end{lemma} \pause
{\bf Note: } $\langle -,-\rangle$ is the so-called bilinear Euler form of a quiver.
\end{frame}

\begin{frame}
    \frametitle{Proof of rigidity - conclusion}
\begin{theorem}

    Under a technical assumption (that $\tuple{d}$ is \emph{strongly amply $\theta$-stable}),
    the Teleman inequality holds on every stratum: for all $\tuple{d}^*$,
    \[\max(k_m - k_n) < \sum_{1 \leq m < n \leq s} (k_n - k_m)\langle \tuple{d}^m, \tuple{d}^n \rangle.\] \pause
\end{theorem}
\begin{corollary}
    
    Applying Teleman quantization, we obtain that for all $k \geq 1$,
    \[ \HH^k(\modulispace^{\theta\stable}(Q, \tuple{d}), \mathcal{U}^{\vee}_i \otimes \mathcal{U}_j) = 0. \]
    This concludes the proof. \qed \pause
\end{corollary}
\end{frame}
\begin{frame}
    \frametitle{Conclusions}
\begin{itemize}
\item Quiver moduli are a large class of examples of interesting varieties; \pause
\item They can easily be constructed to be smooth, projective, Fano... \pause 
\item Many hard, abstract questions are reduced to recursive, implementable problems. \pause
\item In fact, these are implemented :)
\end{itemize}
\end{frame}
\begin{frame}
\begin{center}
    Thank you for your attention!
\end{center}
\end{frame}

\begin{frame}
    \frametitle{Bibliography}
    \printbibliography
\end{frame}

\end{document}